\documentclass[12pt, journal, a4paper, onecolumn]{IEEEtran}

\usepackage[utf8]{inputenc}
\usepackage[spanish, es-tabla, es-nodecimaldot]{babel}
\usepackage{amsmath}
\usepackage{natbib}
\usepackage{amsfonts}
\usepackage{amssymb}
\usepackage{graphicx}
\author{Martin Alejandro Paredes Sosa}
\title{Empaquetamiento de esferas duras}
\date{Febrero 2018}

\begin{document}
\maketitle

\section*{Close Packing}
El close 

\section*{Random Close Packing}
La nocion del Random Close Packin (RCP) es la densisdad máxima que una gran colección de esferas puede obtener y que esta densisdad es una cantidad universal.

\nocite{*}
\bibliographystyle{plain}
\bibliography{biblio}


\end{document}