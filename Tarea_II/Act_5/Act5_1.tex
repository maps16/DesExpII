\documentclass[12pt, journal, a4paper, onecolumn]{IEEEtran}

\usepackage[utf8]{inputenc}
\usepackage[spanish, es-tabla, es-nodecimaldot]{babel}
\usepackage{amsmath}
%\usepackage{natbib}
\usepackage{amsfonts}
\usepackage{amssymb}
\usepackage{graphicx}
\author{Martin Alejandro Paredes Sosa}
\title{Empaquetamiento de Esferas Duras}
\date{Febrero 2018}

\begin{document}
\maketitle

\section*{Close Packing}
El close packing, es el arreglo de un número infinito de esferas dentro de un espacio infinito tridimensional, de tal forma que estas ocupen el mayor espacio posible. Carl Friedrich mostro que la mayor densidad media que se puede alcanzar es \cite{wikipedia_2018}

\begin{equation}
	\frac{\pi}{3\sqrt{2}} = \approx 0.74048
	\label{Dens}
\end{equation}
La conjeturta de Kepler establece que la mayor densidad posible para cualquier arreglo de esferas iguales es $\eta = \pi/(3\sqrt{2})$ \cite{eric}.








\section*{Random Close Packing}
La nocion del Random Close Packin (RCP) es la densisdad máxima que una gran colección de esferas puede obtener y que esta densisdad es una cantidad universal. \cite{torquato} Este parametro se define de manera estadística y los resultados son empíricos.























\nocite{*}
\bibliographystyle{IEEEtran}
\bibliography{biblio}


\end{document}