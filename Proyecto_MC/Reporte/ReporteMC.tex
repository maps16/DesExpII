\documentclass[12pt,letterpaper]{article}

\usepackage[spanish, es-tabla, es-nodecimaldot]{babel}
\usepackage[utf8x]{inputenc}
\usepackage{amsmath}

%\usepackage{hyperref}
%\usepackage{url}
\usepackage{textcomp}
\usepackage{gensymb}
\usepackage[dvipsnames]{xcolor}

\usepackage{parskip}
\usepackage{fancyhdr}
\usepackage{multicol}
\usepackage{vmargin}
\usepackage{setspace}
\usepackage{geometry}

\usepackage{float}
\usepackage{array}
\usepackage{graphicx}
\graphicspath{{Images/}}
\usepackage{wrapfig}
\usepackage{caption}
\usepackage{subcaption}
\usepackage[square, numbers, sort]{natbib}

\usepackage{listings}
\usepackage{color}
%\usepackage[usenames,dvipsnames]{color}
	\definecolor{ocre}{RGB}{42,105,21}
	\definecolor{ocre2}{RGB}{0,102,0}%47,109,130}
	\definecolor{gray2}{gray}{0.95}
	\lstset{
		language={[03]fortran},
		backgroundcolor=\color{gray2},
		basicstyle=\color{black}\small\ttfamily, 
		breakatwhitespace=false,         
		breaklines=true,                 
		captionpos=b,                    
		columns=flexible,
		commentstyle=\color{ocre2}\ttfamily, 
		deletekeywords={...},            
		escapeinside={\%*}{*)},          
		extendedchars=true,             
		frame=single,	                 
		keepspaces=true,                 
		keywordstyle=\color{blue}\bfseries,       
		otherkeywords={*,...},          
		numbers=left,                    
		numbersep=5pt,                   
		numberstyle=\small, 
		rulecolor=\color{black},         
		showspaces=false,                
		showstringspaces=false,          
		showtabs=false,                  
		stepnumber=1,                    
		stringstyle=\normalfont\color{ocre},     
		tabsize=2,	                     
		title=\lstname                  
		}
\definecolor{labelcolor}{RGB}{100,0,0}



\setmarginsrb{2.0cm}{1.0cm}{2.0cm}{2.5cm}{0.5cm}{1cm}{1 cm}{1 cm} %{izq}{up}{der}{down}{Encabezado}

\pagestyle{fancy}
\fancyhf{}
\rhead{Lic. Física}
\lhead{Desarrollo Experimental II}
\cfoot{\thepage}


\title{ Comentarios, Desarrollos u Observaciones  }

\newcommand{\pd}[3] {\left(\frac{\partial #1}{\partial #2}\right)_{#3}}
\newcommand{\pdd}[3] {\left(\frac{\partial^2 #1}{\partial {#2}^2}\right)_{#3}}


\begin{document}


\begin{titlepage}
	\centering
    \vspace*{2cm}
	{\Large Proyecto Final \par}
	\vfill
	{\Large Desarrollo Experimental II \par}
	\vfill
	{\large\ Docente:\\ Dra. Laura Lorenia Yeomans Reyna \par}
    \vfill
    {\large\ \textbf{Simulación de Monte Carlo:}\\ 
    \emph{Exploración de la Ecuación de Van der Waals}\par}
    \vfill
    {\large\ Alumno:\\ Martín Alejandro Paredes Sosa \par}
	\vfill
	% Bottom of the page
	{\large Semestre: 2018-1\par}
\end{titlepage}

\section{Introducción}
	
	La estructura de un líquido está fuertemente determinada por las interacciones repulsivas de corto alcance, y  los estudios con simulaciones computacionales se han utilizado para mostrar que estas interacciones repulsivas de corto alcance pueden modelarse como de núcleo duro. Para modelar un líquido, sin embargo, se requiere una componente atractiva como parte del potencial de interacción entre las partículas del sistema.  

Una de las ecuaciones de estado más importantes y que históricamente se han plateado para describir a los líquidos y la coexistencia líquido-vapor ha sido la ecuación de estado de Van der Waals \cite{Modern}:

\begin{equation}
	\left( p + \frac{N^2}{V^2}a \right) \left( V -Nb \right) = Nk_B T
	\label{VanDerWaals}
\end{equation}

Esta se puede derivarsre como una aplicación de la teoría de perturbaciones de Zwanzig\cite{Stats}, que permite obtener las expresiones para los coeficientes $a$ y $b$ que coinciden con la propuesta de Van Der Waals \cite{Stats} 

\begin{align}
	a &= -2\pi \int_{\sigma}^{\infty} u^{(1)}(r)r^2dr \label{Coef_A_1} \\
	b &= \frac{2\pi\sigma^3}{3} \label{Coef_B_1}
\end{align}
donde $\sigma$ es el diametro de la esfera dura y $u^{(1)}(r)$ el potecial de interacción atractivo que se considera como la parate perturbativa del potencial de interacción entre las partículas del sistema. 

\begin{equation}
u(r) = u^{(hs)}(r) + u^{(1)}(r)
\label{PotInter_Base}
\end{equation}
\pagebreak

\section{Contexto}
Para la obtención de \eqref{Coef_A_1}, se asume que:
\begin{equation}
	 g_{hs}\approx
		\left\{
		\begin{aligned}
        	0 & \qquad r\leq \sigma\\
	        1 & \qquad r > \sigma ,
       \end{aligned}
       \right.
       \label{gdrhs}
\end{equation}

Esta aproximación es para muy bajas concentraciones, donde la función de distribución radial de contacto $g_{hs}(\sigma^+) \approx 1$, que corresponde al modelo de Van Der Waals.

En el contexto del ensemble canónico en la Física Estadística, la ecuación de estado se obtiene a partir de:
\begin{equation}
	p = \frac{1}{\beta} \left( \frac{\partial Z_N}{\partial V} \right)_{T}
	\label{EnsembleP}
\end{equation}
Como $g_{hs}(r)$ depende de la concentración, la ecuacion de estado es de la forma:

\begin{equation}
	p = p_{hs} - \rho^2\left[ a(\rho) + rho \left( \frac{\partial a(\rho)}{\partial\rho} \right)_T \right] 
	\label{Press}
\end{equation}
\begin{equation}
	p_{hs} = \rho k_B T\left[ 1 +\frac{2}{3}\pi\sigma^3\rho g_{hs}(\sigma^+) \right]
	\label{PressHS}
\end{equation}

\pagebreak


\section{Metodología}
\begin{enumerate}
\item[I.]\textbf{Potencial Perturbativo:} La teoria de Zwangzig no impone ninguna restricción sobre el potencial perturbativo $u^{(1)}(r)$ con tal de que sea atractivo. Entonces, consideremos  como potencial  perturbativo la parte atrativa del modelo de pozo cuadrado:
\begin{equation}
	u^{(1)}(r) =
	\left\{
	\begin{aligned}
        -\varepsilon &  \qquad \sigma < r < \lambda\sigma\\
		0 & \qquad r \geq \lambda\sigma 
    \end{aligned}
    \right.
	\label{Pozo}
\end{equation}

De modo que el potecial de interacción segun la ecuación \eqref{PotInter_Base} queda:

\begin{equation}
	u(r)=
	\left\{
	\begin{aligned}
		\infty & \qquad r \leq \sigma \\
        -\varepsilon &  \qquad \sigma < r < \lambda\sigma\\
		0 & \qquad r \geq \lambda\sigma \\
	\end{aligned}
    \right.
	\label{PotInteracTotal}
\end{equation}

\item[II.] \textbf{Reducción de variables} Antes de empezar ha realizar calculos, se redujeron las ecuaciones, tomando como longitud caracteristica el diametro $\sigma$ y como energía caracteística la energía térmica $\beta^{-1}$

Por lo tanto obtenemos las siguientes definiciones:
\begin{align}
	r^* &\equiv \frac{r}{\sigma} \label{rRed} \\
	u^*(r) &\equiv \frac{u(r)}{k_B T} = \beta u(r) \label{potRed}\\
	p^* &\equiv \frac{p\sigma^3}{k_B T} =\beta p\sigma^3 \label{presRed}\\
	T^* &\equiv \frac{k_B T}{\varepsilon} =\frac{1}{\beta \varepsilon} = \frac{1}{\varepsilon^*} \label{TempRed} \\
	n^* &\equiv \sigma^3\rho \label{ConsRed}
\end{align}
 De tal forma que:
\begin{equation}
	u^*(r)=
	\left\{
	\begin{aligned}
		\infty & \qquad r^* \leq 1 \\
        \frac{-1}{T^*} &  \qquad 1 < r^* < \lambda\\
		0 & \qquad r^* \geq \lambda \\
	\end{aligned}
    \right.
	\label{PotInteracTotal_Red}
\end{equation}


Utilizando \eqref{rRed}-\eqref{ConsRed} en \eqref{PressHS} obtenemos:
\begin{align}
\frac{p^*_{hs}}{\beta\sigma^3} &= \frac{n^*}{\beta\sigma^3}\left[ 1 +\frac{2}{3}\pi\sigma^3\frac{n^*}{\sigma^3} g_{hs}(1^+) \right] \nonumber \\
p^*_{hs} &= n^* \left[ 1 +\frac{2}{3}\pi n^* g_{hs}(1^+) \right] \nonumber \\
p^*_{hs} &= n^* \left[ 1 + n^*b^*\right] 
\intertext{Donde:}
b^* &= \frac{2}{3}\pi g_{hs}(1^+)
\end{align}
Ahora en \eqref{Press} obtenemos:

\begin{align}
\frac{p^*}{\beta\sigma^3} &= \frac{p^*_{hs}}{\beta\sigma^3} - \left(\frac{n^*}{\sigma^3}\right)^2 \left[ a(n^*) + \frac{n^*}{\sigma^3} \left( \frac{\partial a(n^*)}{\partial\frac{n^*}{\sigma^3}} \right)_{T^*} \right] \nonumber \\
p^* &= p^*_{hs}  - \frac{\beta {n^*}^2}{\sigma^3}\left[ a(n^*) + n^* \left( \frac{\partial a(n^*)}{\partial n^*} \right)_{T^*} \right] \nonumber \\
p^* &= p^*_{hs} - {n^*}^2\left[ \frac{\beta}{\sigma^3}a(n^*) + n^* \left(\frac{\partial\frac{\beta}{\sigma^3}a(n^*)}{\partial n^*}\right)_{T^*}\right] \nonumber \\
p^* &= p^*_{hs} - \left[ a^* + n^* \left(\frac{\partial a^*}{\partial n^*}\right)_{T^*} \right] {n^*}^2
\intertext{Donde:}
a^* &= \frac{\beta}{\sigma^3}a(n^*)
\end{align}

\item[III.] \textbf{Parametro Críticos} 
La Ecuación de Van Der Waals \eqref{VanDerWaals} tiene una temperatura crítica $T_c$.
Caundo $T>T_c$ la ecuación de estado es monovaluada y no hay transiciones al estado liquido. Caundo $T<T_c$, como la ecuación es cubica en $V$, tiene dos extremales que se juntan en $T=T_c$.

Pensando en la ecuación de Van Der Waals como una fución del volumen a una temperatura dada $P = P(V;T)$, se tienen dos punto críticos donde se cumple   $\left( \partial_V P \right)_T = 0 $. Conforme $T\rightarrow T_c$ se obtiene un punto de inflexión con coordenadas $(P_c, V_c, T_c)$ donde $\left( \partial^2_V P \right)_T=0$

Partiendo de la ecuación \eqref{VanDerWaals}:
\begin{align}
	P &= \frac{Nk_B T}{V-Nb} - \frac{N^2}{V^2}a	\label{PresVanDerWaals} \\
	\pd{P}{V}{T} &= -\frac{Nk_BT}{(V-Nb)^2} + \frac{2N^2a}{V^3} \\
	\pdd{P}{V}{T} &= \frac{2Nk_BT}{(V-Nb)^3} - \frac{6N^2a}{V^4}
\end{align}




\end{enumerate}



\pagebreak










\bibliographystyle{plain}
\bibliography{biblio}


\end{document}

