
\documentclass[12pt,letterpaper]{article}

\usepackage[spanish, es-tabla, es-nodecimaldot]{babel}
\usepackage[utf8x]{inputenc}
\usepackage{amsmath}

\usepackage{hyperref}
\usepackage{url}
\usepackage{textcomp}
\usepackage{gensymb}
\usepackage[dvipsnames]{xcolor}

\usepackage{parskip}
\usepackage{fancyhdr}
\usepackage{multicol}
\usepackage{vmargin}
\usepackage{setspace}
\usepackage{geometry}

\usepackage{float}
\usepackage{array}
\usepackage{graphicx}
%\graphicspath{{images/}}
\usepackage{wrapfig}
\usepackage{caption}
\usepackage{subcaption}

\setmarginsrb{2.0cm}{1.0cm}{2.0cm}{2.5cm}{0.5cm}{1cm}{1 cm}{1 cm} %{izq}{up}{der}{down}{Encabezado}

\pagestyle{fancy}
\fancyhf{}
\rhead{Lic. Física}
%\lhead{\thesection}
\cfoot{\thepage}


\title{ Comentarios, Desarrollos u Observaciones  }

\begin{document}


\begin{titlepage}
	\centering
    \vspace*{2cm}
	{\Huge Comentarios, Desarrollos u Observaciones \par}
	\vfill
	{\Large Desarrollo Experimental II \par}
	\vfill
	{\large\ Docente: Dra. Laura Lorenia Yeomans Reyna \par}
    \vfill
    {\large\ \textbf{Tarea I}:\\ Ejercicios para poner a punto compiladores y graficadores \par}
    \vfill
    {\large\ Martín Alejandro Paredes Sosa \par}
	\vfill
	% Bottom of the page
	{\large Semestre: 2018-1\par}
\end{titlepage}
\section{Tarea I}
A continuacion se muestran  los cometario relacionados con la primer parte del Portafolio, la Tarea I.
\vspace{-0.5cm}

\subsection*{Actividad 1}
La actividad 1 que consistió en graficar una función resultó sencilla, solo ocupe refrescar algunas ideas en la declaración de variables y la declaración de la función.

\subsection*{Actividad 2}
Esta consistio en generar puntos sobre una recta alternando entre positivos y negativos cuando era imapar o par el punto a colocar.

En esta actividad me tome mi tiempo. Experimeten como diferentes formas de realizar este y logre realizarlo de dos formas diferentes. Una mas compacta que la segunda (con ayuda).

Un reto es entender que es muy util tener la mitad de la distancia en las fronteras.

\subsection*{Actividad 3}
Fue sencillo de realizar, solo que esta vez tuve que lidear con N impar ya que en un principio tenia mas particulas de un lado. Resolvi este problema con condiones para las N impar.

\pagebreak

\section{Tarea II}
A continuacion se muestran  los cometario relacionados con la segunda parte del Portafolio, la Tarea II.
\vspace{-0.5cm}

\subsection*{Actividad 1}
Es un problema sencillo con lo que nos dice en clase, ademas permite trabajar los condicionales combinados en los ciclos. Lo dificil es acordarse de lo básico.


\subsection*{Actividad 2}
Este codigo fue complicado debido a que fue la primer vez que utilice un generador de numeros aleatorios. En un principio me confundió debido a que no vi diferencia en los resultados cuando lo corri multiples veces. Luego entendi que con el uso de los \textit{seeds} podia generar diferentes resultados aleatorios.


\subsection*{Actividad 3}
Este codigo fue sencillo ya que solo se debe modificar algunas lineas del codigo para reducir de 3 a 2 dimensiones. Tal vez esta actividad es algo inesecaria pero util para entender el trabajo con diferente dimensiones.



\subsection*{Actividad 4}
Lo complicado fue decidir cuanta particulas para calocarlas de manera regular. Pero fuera de eso, fue sencillo adaptar el codigo de las actividades pasadas.


\subsection*{Actividad 5}
La informacion de este tema, esta muy perdida y se parece mucho entre las difernetes fuentes. 


\end{document}
