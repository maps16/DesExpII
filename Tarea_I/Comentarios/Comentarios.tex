
\documentclass[12pt,letterpaper]{article}

\usepackage[spanish, es-tabla, es-nodecimaldot]{babel}
\usepackage[utf8x]{inputenc}
\usepackage{amsmath}

\usepackage{hyperref}
\usepackage{url}
\usepackage{textcomp}
\usepackage{gensymb}
\usepackage[dvipsnames]{xcolor}

\usepackage{parskip}
\usepackage{fancyhdr}
\usepackage{multicol}
\usepackage{vmargin}
\usepackage{setspace}
\usepackage{geometry}

\usepackage{float}
\usepackage{array}
\usepackage{graphicx}
%\graphicspath{{images/}}
\usepackage{wrapfig}
\usepackage{caption}
\usepackage{subcaption}

\setmarginsrb{2.0cm}{1.0cm}{2.0cm}{2.5cm}{0.5cm}{1cm}{1 cm}{1 cm} %{izq}{up}{der}{down}{Encabezado}

\pagestyle{fancy}
\fancyhf{}
\rhead{Lic. Física}
%\lhead{\thesection}
\cfoot{\thepage}


\title{ Comentarios, Desarrollos u Observaciones  }

\begin{document}


\begin{titlepage}
	\centering
    \vspace*{2cm}
	{\Huge Comentarios, Desarrollos u Observaciones \par}
	\vfill
	{\Large Desarrollo Experimental II \par}
	\vfill
	{\large\ Docente: Dra. Laura Lorenia Yeomans Reyna \par}
    \vfill
    {\large\ \textbf{Tarea I}:\\ Ejercicios para poner a punto compiladores y graficadores \par}
    \vfill
    {\large\ Martín Alejandro Paredes Sosa \par}
	\vfill
	% Bottom of the page
	{\large Semestre: 2018-1\par}
\end{titlepage}
\section*{Actividad 1}
La actividad 1 que consistió en graficar una función resultó sencilla, solo ocupe refrescar algunas ideas en la declaración de variables y la declaración de la función.

\section{Actividad 2}
Esta consistio en generar puntos sobre una recta alternando entre positivos y negativos cuando era imapar o par el punto a colocar.

En esta actividad me tome mi tiempo. Experimeten como diferentes formas de realizar este y logre realizarlo de dos formas diferentes. Una mas compacta que la segunda (con ayuda).

Un reto es entender que es muy util tener la mitad de la distancia en las fronteras.

\section{Actividad 3}
Fue sencillo de realizar, solo que esta vez tuve que lidear con N impar ya que en un principio tenia mas particulas de un lado. Resolvi este problema con condiones para las N impar.
\end{document}
